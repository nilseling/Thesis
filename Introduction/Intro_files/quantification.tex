%!TEX root = ../intro.tex
%******************************
%	 Quantification of biological noise
%*****************************

\section{Quantification of biological noise} 

To quantify biological noise in cellular systems, single cell approaches employ either sequencing or imaging technologies to extract genomic, transcriptomic, epigenomic, metabolomic or proteomic information. These technologies show specific advantages and limitations on the level of throughput and content.

\subsection{Single-cell sequencing}
Next generation sequencing approaches of individual cells are employed to quantify variation in DNA sequence, mRNA expression, epigenetic marks and protein abundance within a cell population. 

\subsubsection{Single-cell whole genome sequencing}
Single cell whole genome sequencing has previously been used to identify copy number variation and single-nucleotide polymorphisms between single cells(Shpunt et al., 2012). From these information, tumor heterogeneity and evolution(Navin et al., 2011) as well as lineage relationship in the human brain were inferred(Evrony et al., 2015). To obtain enough genomic material, whole genome amplification is performed on DNA from individual cells. The single cell comparative genomic hybridization protocol (SCOMP) degrades DNA via restriction enzymes, includes a primary PCR amplification step and a later re-amplification via comparative genomic hybridization(Klein et al., 1999). Multiple displacement amplification (MDA) is based on the random initiation of amplification via oligonucleotide primers with strand displacement(Dean et al., 2002). Compared to MDA, multiple annealing and looping-based amplification cycles (MALBAC) achieve an initial quasi-linear amplification step by pre-amplification using primers with handle sequences. Full amplicons form hairpins that are exponentially amplified prior to sequencing(Shpunt et al., 2012). 

\subsubsection{Single-cell RNA sequencing}
Initial approaches to quantify mRNA abundance within single cells (Single cell RNA sequencing) included targeted microfluidic-based single cell RT-PCR(Warren et al., 2006) and whole-transcriptome read-outs of hand-picked cells(Tang et al., 2009). Methods of cell capture range from micromanipulation(Rashel V. Grindberg et al., 2014) and laser capture microdissections(Frumkin et al., 2008) as targeted methods with low throughput to  fluorescence-activated cell sorting (FACS)(Hayashi et al., 2010; Dalerba et al., 2011; Jaitin et al., 2014), microfluidics(Trapnell et al., 2014; Treutlein et al., 2014) and microdroplets(Klein et al., 2015; Macosko et al., 2015) as high-throughput approaches. More broadly, single-cell RNA sequencing (scRNAseq) approaches can be grouped into valve-, droplet- or well-based strategies(Prakadan, Shalek and Weitz, 2017).\\
A variety of scRNAseq protocols have been published that utilize different methods for mRNA reverse transcription, cDNA amplification and library preparation. All commonly used techniques for scRNAseq select and reverse transcribe mRNA (poly[A] tailing). The initial protocol introduced by Tang et al.(Tang et al., 2009) was improved due to incorporating a template switching mechanism at the 5' end of the mRNA thus reducing the 3' sequencing bias present in previous methods(Islam et al., 2011). This STRT method was later modified for full-length transcript detection (SmartSeq(Ramsköld et al., 2012) and SmartSeq2(Picelli et al., 2013)). CELseq(Hashimshony et al., 2012) and CELseq2(Hashimshony et al., 2016) uses in vitro transcription (IVT) to linearly amplify cDNA prior to sequencing. Current protocols for sequencing library preparation are optimized for Illumina, SOLiD or PacBio sequencing(Aleksandra A. Kolodziejczyk et al., 2015). 
During scRNAseq minute amounts of mRNA are captured and amplified generating a high degree of technical noise, which distorts quantification of true biological variability. To account for this, a set of external RNAs (ERCC spike-ins) developed by the External RNA Control Consortium(External RNA Controls Consortium, 2005) can be added to the cell lysate. Based on the reads mapped to ERCC spike-ins, technical noise can be removed from total expression variability(Brennecke et al., 2013; Vallejos, Marioni and Richardson, 2015). Another way to reduce noise derived from amplification biases in scRNAseq experiments is to tag each mRNA molecule with a unique molecular identifier (UMI)(Kivioja et al., 2011; Islam et al., 2014). 
Droplet-based scRNAseq technologies increase scalability for scRNAseq studies. Individual cells are captured in nanoliter-sized droplets where RT is performed and transcripts are marked with cell-specific barcodes and UMIs(Klein et al., 2015; Macosko et al., 2015). These approaches form a basis for characterizing large scale differentiation processes(Ibarra-Soria et al., 2018) or for generating a map of all human cell types(Regev et al., 2017).

\subsubsection{Single-cell epigenome sequencing}

Single cell epigenomic methods capture the chromatin and DNA methylation state of individual cells and allow quantification of epigenetic variability across a population of cells (see below)(Clark et al., 2016). To observe methylation states of CpG motifs on the DNA, single cell bisulfite sequencing (scBSseq) includes the extraction of genomic DNA and bisulfite conversion prior to library preparation(Smallwood et al., 2014; Farlik et al., 2015). Single cell reduced representation bisulfite sequencing (scRRBSseq) first enzymatically digests genomic DNA prior to bisulfite conversion. CpG-rich fragments can be enriched and amplified via ligated adapters before high-throughput sequencing(Guo et al., 2013). Extending the read-out of scBSseq and scRRBSseq, single cell 5-hydroxymethylcytosine sequencing (5hmCseq) captures the first oxidative product of CpG sites towards de-methylation and therefore cellular variation of methylation dynamics. Instead of bisulfite conversion, 5hmC sites are glucosylated before enzymatic digestion and adapter ligation(Mooijman et al., 2016).
To measure histone modifications or transcription factor binding dynamics on the single-cell level, digested chromatin from individual cells is tagged with barcodes prior to immunoprecipitation (IP) during single-cell chromatin IP followed by sequencing (scChIPseq). With this droplet-based method variable chromatin signatures were detected across a population of embryonic stem cells based on the H3K4me2 histone modifcation(Rotem et al., 2015). 
Other approaches focus on estimating the patterns of open chromatin by assessing chromatin accessibility. Single-cell assay of transposase-accessible chromatin using sequencing (scATACseq) captures individual cells on integrated fluidics circuits (IFC) before inserting sequencing adapters into accessible regions via the prokaryotic Tn5 transposase and pre-amplifiation. After library collection, cell-specific barcodes were added via a second round of PCR prior to sequencing(Buenrostro et al., 2015).  Capturing cells in IFCs before barcoding limits the throughput to around tens or hundreds of cells at one time. Combinatorial indexing by tagging cells with barcodes in a two step process increases throughput for scATACseq to thousands of cells(Cusanovich et al., 2015). An alternative approach to measure open chromatin involves the digestion of DNA with DNase I (PicoSeq). The resulting small fragments undergo end-repair, adaptor ligation and PCR amplification in the presence of circular carrier DNA to avoid the loss of minute amount of fragments(Jin et al., 2015). Similarly, nucleosome positioning can be detected by using the GpC-specific DNA methyltransferase (MTase) M.CviPI to methylate cytosines of GpC motifs in regions where DNA is accessible. Individual cells were isolated and their DNA digested prior to bisulfite conversion. Patterns of methylated and unmethylated GpCs indicate the positioning of nucleosomes along the DNA(Small et al., 2014).
Single-cell technologies to study large-scale chromosome structure include DamID(Kind et al., 2015), a method to identify lamina-associated domains, and single-cell HiC(Nagano et al., 2013).

\subsubsection{Multi-omics approaches}

In recent years, the above described techniques have been combined to measure transcriptomic, genomic, epigenomic and proteomic (“multi-omic”) features of single cells in parallel(Macaulay, Ponting and Voet, 2017).
The first approach to sequence DNA and mRNA (DR-seq) from the same cell amplifies genomic DNA and cDNA derived from reverse transcribed mRNA in one reaction step to avoid losses. After initial amplification, the sample is split to further process gDNA and cDNA separately. PCR amplification increases the amount of gDNA while IVT amplifies cDNA prior to sequencing(Siddharth S. Dey et al., 2015). An alternative approach (G\&{}T-seq) firstly separates gDNA and mRNA before whole-transcriptome and whole-genome amplification. Biotinylated oligo-dT primer captures mRNA and is coupled to streptavidin coated beads. Once mRNA and gDNA is separated, the SmartSeq2 protocol performs whole-transcriptome amplification while MDA or PicoPlex approaches can be used to amplify gDNA prior to sequencing(Macaulay et al., 2015).
Similarly, single cell methylome and transcriptome sequencing (scM\&{}T-seq) initially separates genomic DNA from mRNA. The scBSseq protocol was applied to isolated gDNA to identify methylated CpG positions while mRNA was amplified via the SmartSeq2 protocol(Angermueller et al., 2016).   
The scM\&{}T-seq method has been extended to detect accessible chromatin regions in parallel to capturing methylated CpG sites and whole-transcriptome information. Prior to bisulfite conversion of gDNA, GpC sites are methylated by MTase in nucleosome spares regions(Pott, 2017; Clark et al., 2018). 
Attempts have been made to capture ~96 mRNAs in combination with proteins within individual cells. After cell lysis, samples are split to process mRNA and protein separately. mRNA is reverse transcribed and pre-amplified prior to qPCR while oligonucleotide tagged antibodies bind to proteins. The free 3’-ends are complementary and can be extended by polymerization to create a DNA reporter molecule. Similar to mRNAs, these molecules are detected using qPCR(Darmanis et al., 2016). This method has been scaled up by integration of droplet digital PCR(Albayrak et al., 2016). Alternatively, proximity ligation assay for RNA (PLAYR) allows isotope tagging of RNA molecules, which are detected in parallel to proteins via mass cytometry(Frei et al., 2016).
Imaging approaches
Similar to single cell sequencing, RNA or protein imaging approaches quantify noise in biological systems(Harton and Batchelor, 2017). Initial studies that addressed the extend of biological noise in bacterial populations used the expression of fluorescent proteins to quantify expression noise(Elowitz et al., 2002; Blake et al., 2003). Later on, single molecule fluorescence in situ hybridization (smFISH) was developed to capture variation in mRNA abundance across multiple cells(Fang et al., 2013; Lyubimova et al., 2013; Sanchez and Golding, 2013). Furthermore, the combination of fluorescently labeled proteins and single molecule fish allow the detection of co-variation between protein and mRNA levels within individual cells(Taniguchi et al., 2010). High-throughput automated smFISH of target RNAs in thousands of wells(Battich, Stoeger and Pelkmans, 2013) identified nuclear retention of RNAs as control mechanism to reduce cytoplasmic transcript variability(Battich, Stoeger and Pelkmans, 2015). Moreover, computerized image analysis and supervised machine learning extracts hundreds of cellular features from microscopy images and can therefore dissect variation of biological processes as virus infection(Snijder et al., 2009).
The development of super-resolution microscopy allows detection of fluorophores that are spaced less than 100nm apart(Sydor et al., 2015). By combining stochastic optical reconstruction microscopy (STORM) and combinatorial labeling of RNA inside the cell, multiple transcripts from different genes can be visualized(Lubeck and Cai, 2012). This approach has been advanced to measure hundreds to thousands of RNA species per cells. Multiplexed error-robust fluorescence in situ hybridization (MERFISH) hybridizes encoding probes to target RNAs prior to N rounds of combinatorial labeling using fluorescently labeled read-out probes. MERFISH uses an encoding scheme that corrects for individual read-out errors based on a certain hamming distance between possible N-bit codes. Therefore, with 16 rounds of combinatorial labeling and a hamming distance of 4, 140 RNA species can be detected(Chen et al., 2015). By replacing the photobleaching step between consecutive rounds of combinatorial labeling with chemical cleavage and using multi-color imaging, the throughput of MERFISH can be increased(Moffitt, Hao, Wang, et al., 2016). Background fluorescence in tissue sections can be reduced by matrix-embedding of labeled RNA and cellular digestion(Moffitt, Hao, Bambah-Mukku, et al., 2016).
Describe the papers from Long Cai lab once I can access them(Yang et al., 2014; Coskun and Cai, 2016; Shah et al., 2016; Eng et al., 2017; Frieda et al., 2017)


\subsection{Imaging approaches}

\subsection{Other approaches}

\subsection{Computation modelling and quantification}

%The mathematical and computational framework to model transcriptional and translation dynamics in biological systems was focus of previous in depth research(Tsimring 2014). In the simplest case, the central dogma of molecular biology defines that mRNAs are synthesized from DNA at rate k_m and proteins are translated from mRNAs at rate k_p. Furthermore, mRNAs are degraded at rate m and proteins at rate p. In a noise-free system, this dogma leads to the following deterministic, first-order differential equation describing the mRNA and protein dynamics:

%dm/dt  = k_m-γ_m m,   dp/dt  = k_p m-γ_p p

%Steady-state transcript count in this simple, two-stage system is defined as 〈m〉=k_m/γ_m  and protein abundance as 〈p〉=(k_m k_p)/(γ_m γ_p ). The second moments for transcript and protein distributions are defined as: σ_m^2=〈m〉 and σ_p^2=〈p〉[k_p/(γ_p+γ_m )+1]=〈p〉[b/(1+η)+1],
%where b=k_p/γ_m  is the average number of proteins produced per transcript and η=γ_p/γ_m  (Tsimring 2014; Thattai \& van Oudenaarden 2001). mRNAs usually decay much faster than proteins. Therefore γ_m≫γ_p and σ_p^2≅〈p〉[b+1] (Thattai \& van Oudenaarden 2001).
%For this system, the mean translational burst size can be described as the Fano factor (σ_p^2)/〈p〉 ≅b+1≈b and burst frequency is captured by the inverse squared coefficient of variation 〈p〉^2/(σ_p^2 )≈〈p〉/b=k_m/γ_p =a. The latter assumes that mRNAs are directly translated as soon as they are produced (Friedman et al. 2006).
%To account for stochasticity in this system, probabilistic expressions of aforementioned equations have been described. The chemical master equation defines the time-evolution of the probability of observing a system containing m mRNAs and p proteins at time-point t:

%(∂P_(m,p))/∂t=k_n [P_(m-1,p)-P_(m,p) ]+γ_m [(m+1)P_(m+1,p)-mP_(m,p) ]
%+k_p m[P_(m,p-1)-P_(m,p) ]+γ_p [(p+1)P_(m,p+1)-pP_(m,p) ]

%The stationary probability distribution for this discrete representation of the master equation has the form of a negative-binomial distribution:

%P_p=Γ(a+n)/Γ(n+1)Γ(a)  (b/(1+b))^n (1-b/(1+b))^a

%where a represents the burst frequency, b the mean burst size and Γ(n) the Gamma function(Shahrezaei \& Swain 2008). Friedman et al. derived a stationary probability distribution from a continues form of the chemical master equation(Friedman et al. 2006). This solution takes the form of a Gamma distribution:

%P_p=1/(b^a Γ(a)) n^(a-1) e^(-n/b)

%This simple system has been extended to incorporate the ON-OFF switching of promoters(Jones et al. 2014; Shahrezaei \& Swain 2008). Extensive modeling and quantification of mRNA and protein abundance in prokaryotic and eukaryotic cell populations confirmed this negative binomial (over-dispersed Poissonian) relationship between protein variance and abundance(Ozbudak et al. 2002; Bar-Even et al. 2006). The over-dispersion in protein abundance arises from biological noise (η_tot), which can be decomposed into intrinsic (η_int) and extrinsic (η_ext) contributions (η_tot=η_int+η_ext)(Swain et al. 2002; Fu \& Pachter 2016). These components can be directly computed when using a two reporter system controlled by identical promoters(Elowitz et al. 2002). 
%Classic mathematical approaches to model transcriptional and translational dynamics use simplified assumptions to be solved analytically. Similar to the described translational bursting, transcriptional bursting as observed in eukaryotic cells(Raj et al. 2006) leads to an over-dispersion in mRNA transcripts. Furthermore, while most models focus on single promoter dynamics, cases in which multiple promoters and competitor sites dilute TF binding have only recently been addressed(Das et al. 2015). The assumption that translation from mRNA follows a first-order process was extended by using a hyperbolic, Michaelis-Menten kinetic to model the translation process. This approach allows for continuous levels of ribosome occupancy on mRNAs(Van Dyken 2017). 

%While the models described above theoretically describe the expected distributions of proteins and mRNA across a population of cells, in practice, absolute measures (e.g. transcript counts or fluorescence intensity) have to be used to quantify variation across a population of cells. Previously, a variety of heterogeneity measures were computed to estimate biological noise. 
%In the simplest form, the variance σ^2, either calculated across all cells or across all expressing cells(Shalek et al. 2014a), captures variability in RNA and protein abundance and scales linearly with mean expression μ (S. S. Dey et al. 2015). The squared coefficient of variation or the Fano factor are more widely used to measure heterogeneous RNA expression(Brennecke et al. 2013; Jones et al. 2014) and protein abundance(Newman et al. 2006). Due to technical effects(Brennecke et al. 2013) lowly expressed genes show higher levels of noise compared to highly expressed genes. Therefore, the CV2 decreases with mean expression. To compare variability measures across different biological conditions where mean expression changes, regression approaches have been used to correct for the mean-variance relationship(Aleksandra A. Kolodziejczyk et al. 2015; Fan et al. 2015). 
%Other approaches directly model biological variability as the excess in dispersion after removing technical noise. Similar to the CV2 this over-dispersion measure decreases with increasing mean expression(Vallejos et al. 2015). Moreover, heterogeneous expression can be captured by computing the Shannon entropy. Gene-specific entropy is defined as H=-∑_i▒p_i  〖log〗_2 (p_i) where p_i is the probability for a given gene being expressed in bin i. Binning across the expression counts can be done by choosing a fixed width(Richard et al. 2016) or an adaptive width(Stumpf et al. 2017). Additionally, average pairwise distances between cells capture increasing or decreasing heterogeneity in cell populations(Mohammed et al. 2017). 
I%n some cases, intrinsic variation in gene expression can be masked by large-scale extrinsic effects (e.g. cell being in different cell-cycle stages). Methods have been developed to correct for these confounding effects to dissect otherwise hidden variation(Buettner et al. 2015; Buettner et al. 2017). Describe Loos et al.
