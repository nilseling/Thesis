%!TEX root = ../intro.tex
%******************************
%	 Biological noise 
%*****************************

\section{Biology of expression noise} 

All cellular systems are exposed to varying levels of noise and employ strategies to make use of or cope with this source of variation. 

\subsection{Cell fate decisions}

In light of cell fate decisions, noise facilitates the switch between 
cell states and the probabilistic induction of differentiation processes(Chang et al., 2008; Eldar and Elowitz, 2010). It has been shown that transcriptional noise increases throughout differentiation(Stumpf et al., 2017) and development(Antolović et al., 2017). Dissecting differentiation processes of hematopoetic progenitor cells revealed an increase in transcriptional noise directly before cell fate decisions are made(Mojtahedi et al., 2016; Richard et al., 2016). Once committed, differentiating cell populations collapse in variability and move towards a new attractor state. This process is aided by variable patterns in DNA methylation states locking cells in a terminal differentiated state(Jenkinson et al., 2017). The functional role of noise in stem cells has been widely studied in vivo as well as in vitro:

\subsubsection*{Embryonic development}

Studies of recent years have shown that stochasticity in expression is a crucial driver for early (pre-implantation) embryonic development and prior to gastrulation(Dietrich and Hiiragi, 2007). As early as the 4-cell stage embryo, targets of master pluripotency markers Oct4 (Pou5f1) and Sox2 are heterogeneously expressed.  This is caused by heterogeneous H3R26 methylation patterns induced by CARM1, which in turn facilitates the binding of Oct4 and Sox2 to induce pluripotency. Unmethylated cells differentiate towards the extra-embryonic trophoectoderm while pluripotent cells form the inner cell mass(Goolam et al., 2016). Once the cells compact at the 16-cell stage, transcriptional noise facilitates the correct organization of the inner cell mass and trophoectoderm by improving plasticity(Holmes et al., 2017). In line with this, scRNAseq revealed high levels of noise in the uncommitted inner cell mass at E3.5 (16-cell stage) in comparison to the E4.5 committed epiblast. Noise levels increase again upon exit from pluripotency in the E6.5 epiblast while cells of the primitive streak at E6.5 synchronize their expression patterns and noise is reduced(Mohammed et al., 2017).

\subsubsection*{Stem cell differentiation}

While pluripotent stem cells in the mouse embryo commit irreversibly to cell lineages during development, in vitro cultured mouse embryonic stem cells (mESCs) reside in a self-renewing, metastable state(Hayashi et al., 2008). Transcription factor heterogeneity, especially of the pluripotency regulator Nanog, is highest in LIF/serum grown cells and allows the Nanog-negative cells to commit to differentiation lineages(Chickarmane, Olariu and Peterson, 2012; Torres-Padilla and Chambers, 2014). Heterogeneously expressed genes that show a bimodal distribution in expression counts correlate with each other indicative for the presence of distinct states in mESCs. These distinct states show differences in promoter methylation patterns introducing the role of epigenetic modifications to maintain heterogeneity in mESCs(Singer et al., 2014). In depth analysis of mESCs grown in serum, 2i and a2i media shows the presence of three distinct cell states in the serum grown cells as source of heterogeneity. mESCs grown in 2i media show less variability in pluripotency markers but higher heterogeneity in cell-cycle related genes(Aleksandra A. Kolodziejczyk et al., 2015). From the pluripotent ground state, mESCs can differentiate along somatic lineages via specific differentiation events or noise-induced transition between attractor states. Mathematical modeling has shown that mESCs differentiate stochastically through distinct hidden cell (micro-)states within a defined (macro-)state coupled to an increase in variability(Stumpf et al., 2017).
In contrast to the beneficial features of noise in stem cell differentiation, stochastic events during induced pluripotent stem cell (iPSC) reprogramming limit the formation of single iPSCs(Hanna et al., 2009; Yamanaka, 2009). It has been shown that probabilistic events dominate in an early phase of reprogramming while the transcription of Sox2 induces a later, more deterministic, phase(Buganim et al., 2012).

Another example of stochasticity in cell fate decisions includes commitment of immune cells to Th1 or Th2 lineages. Stochastic expression of lineage defining cytokines Ifng (Th1) and Il4 (Th2) in small populations of cells drive the cell population towards a Th1 or Th2 cell fate while most cells co-express the lineage defining transcription factors Gata3 and Tbx21(Antebi et al., 2013; Fang et al., 2013). Stochasticity in cytokine expression leads to phenotypic variability in the T-helper cell repertoire and increases the effectiveness to respond upon immune stimuli(Schrom and Graham, 2017).
While heterogeneity in gene expression and protein abundance has been characterized to benefit cell commitment and differentiation, other reports propose alternative hypothesis for cell fate decisions where commitment is independent of random fluctuations of transcription factors(Hoppe et al., 2016). Similarly, the increase in noise during development can be counteracted by temporal averaging across noisy transcription events to achieve coordinated tissue responses(Stapel, Zechner and Vastenhouw, 2017). 

\subsection{Tissue development and homeostasis}

Coping with the influence of biological noise is important for regulated tissue development and homeostasis. An early study showed that in order to minimize the effect of stochasticity in development, plants express heat-shock protein 90 to stabilize metastable regulators of growth and development(Queitsch, Sangster and Lindquist, 2002). Furthermore, redundancy in the C. elegans intestinal gene regulatory network buffers variability in down-stream master regulator. Once highly connected regulators of this network are removed, phenotypic variation arises from bimodal expression of the otherwise highly expressed master regulator(Raj et al., 2010). The cooperation of positive and negative feedback loops in these highly connected regulatory networks ensure robust expression of key developmental genes(Ji et al., 2013). While complex signaling networks reduce noise during tissue development, other models have been proposed in which noise helps to form sharp boundaries between neighboring domains(Zhang et al., 2012). While contact based adhesion and repulsion in between cells sharpens narrow transition regions, noise-driven cell state plasticity helps narrowing a wider transition region(Wang et al., 2017). Conversely, rearrangements within a population of cells, allows the correction of sensing errors induced by variation in the strength of single cell responses to a signaling gradient(Camley and Rappel, 2017).
While the cell division rate within tissues is higher during development, tissue homeostasis is maintained by stochastic events that balance cell division and apoptosis(Ranft et al., 2010). The effect of noise on maintaining tissue homeostasis has been studied in a diverse set of organs. In fat tissue, a complex system of signaling feedback loops controls protein abundance noise to induce differentiation at a low rate but prevents stochastic de-differentiation(Ahrends et al., 2014). To maintain coordination in liver function, longer bursts of short lived mRNAs and polyploidy reduce noise in gene expression(Bahar Halpern, Tanami, et al., 2015). Another mechanism to achieve tissue-wide expression responses involves spatial coordination of stochastically expressing cells in the pituitary gland(Featherstone et al., 2016). Spatially constraint signaling events have also been demonstrated to play a role in maintaining colonic crypt cell-type diversity. Per crypt, eight stem cells differentiate into a defined ratio of cell-types. To reduce noise in this process, lateral inhibition within a commitment zone reduces the number of differentiated goblet cells and following slower dispersive migration as well as decreased division rates of goblet cells ensures a distinct 1:3 ratio to enterocytes(Tóth et al., 2017).

\subsection{Evolution}

As discussed above, biological noise shows beneficial features when increased cell plasticity supports switching cell states while in other settings, noise needs to be reduced to allow coordinated expression of cell populations. During evolution, a tradeoff between noise-driven cellular plasticity and robust expression formed. Natural selection acts on genetically controlled expansions of quantitative phenotypes, which are derived from biological noise(Eldar and Elowitz, 2010). Therefore, noisy expression of stress response genes allows a cell population to adapt to changing environments(López-Maury, Marguerat and Bähler, 2009). Specifically, the expression of genes controlled by TATA-box containing promoters shows strong divergence between species(Tirosh et al., 2006). To control for robust expression levels once selection becomes stabilizing, noise levels are reduced(López-Maury, Marguerat and Bähler, 2009; Eldar and Elowitz, 2010; Pires and Conant, 2016). 
In unicellular populations, noise evolutionarily increased as a form of rudimentary regulation(Wolf, Silander and van Nimwegen, 2015). As a consequence, phenotypic heterogeneity increases the adaption rate of cell populations to extreme environments(Bódi et al., 2017). Conversely, in multicellular organisms, collections of cells need to respond in a coordinated manner. It has therefore been proposed that nuclear compartmentalization in higher organisms reduce noise by mRNA retention at the nuclear membrane(Battich, Stoeger and Pelkmans, 2015; Stoeger, Battich and Pelkmans, 2016).

\subsection{Disease}

Biological noise induced phenotypic plasticity within cell populations also plays a role in the onset of diseases. While biological noise supports the adjustment of cells to new microenvironments, errors in form of gene mutations induce transitions from healthy cells towards a cancer attractor state. Cellular plasticity aided by transcriptional noise or conformational variation of intrinsically disordered proteins supports the phenotypic adaption to the new attractor state(Fan et al., 2012). Increased variability in expression can also be observed for more aggressive cancer sub-types across multiple patients(Ecker et al., 2015). In light of cancer treatment, non-genetic cell-to-cell variability in cell populations lead to fractional killing upon exposure to apoptosis-inducing agents(Spencer et al., 2009; Niepel, Spencer and Sorger, 2010; Bertaux et al., 2014). In patient derived melanoma cells, sporadic expression of resistance markers forms a rare cell population that form resistant colonies after treatment. While pre-resistant cells do not display large epigenetic marks and are therefore clos to the non-resistant ground state, treatment induces large epigenetic reprogramming forming stable resistant cancer colonies(Shaffer et al., 2017). 
In the case of HIV treatment it has been shown that combining noise-enhancing and activating drugs shifts latent viruses into the active-replication state that can be targeted by antiretroviral therapeutics(Dar et al., 2014).  This shows that inducing as well as reducing noise in expression are vital strategies for therapy depending on the disease context.
