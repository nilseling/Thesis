%!TEX root = ../chapter3.tex
%******************************
%	 Results 
%*****************************

\section{Droplet based single-cell RNA sequencing of mouse testis}
\subsection*{High-resolution profiling by single-cell RNA-seq captures the unidirectional differentiation of germ cells during adult spermatogenesis}

Spermatogenesis is a recurrent differentiation process that produces male gametes within testicular seminiferous tubules (Fig. 1A). The seminiferous epithelium in the mouse is classified into twelve distinct stages, based on the combination of cell types present (Fig. 1B). Tubules cycle asynchronously and continuously through these stages and adult testis contain tubules in every possible epithelial stage (Fig. 1A and B; Fig. S1A). \\

Using multiple functional genomics approaches, we characterized the transcriptional programme underlying mouse spermatogenesis with cells isolated from specifically staged juvenile (between postnatal days 6 and 35) and adult (8-9 weeks) C57BL/6J (B6) mice. For all samples, we generated unbiased droplet-based single-cell RNA sequencing (scRNA-Seq) data from whole testis with matched histology. Additionally, for juvenile samples, we generated whole-tissue bulk RNA sequencing, as well as mapping chromatin state in purified cell populations using CUT&RUN (Cleavage Under Targets & Release Using Nuclease) (Fig. 1C; Methods) (Skene et al., 2018). After quality control and filtering, we retained a total of 42,796 single cells, 30 bulk RNA-Seq libraries and 8 CUT&RUN libraries (Methods; Table S1).\\

After clustering all single cell transcriptomes using an unbiased graph-based approach (Methods), we first focused on cells isolated from adult B6 testis to generate a comprehensive map of cell types across spermatogenesis. Using computationally-defined cluster-specific marker genes (Methods; Table S2), we identified the following cell types: spermatogonia (based on Dmrt1 expression, Matson et al., 2010), spermatocytes (Piwil1, Deng and Lin, 2002), round and elongating spermatids (Tex21 and Tnp1, respectively, Fujii et al., 2002), as well as the main somatic cell types of the testis, Sertoli (Cldn11, Mazaud-Guittot et al., 2010) and Leydig cells (Fabp3, Oresti et al., 2013) (Fig. 1D). Using a dimensionality reduction technique for visualization (t-distributed Stochastic Neighbour Embedding; Fig. 1E), the germ cell types from spermatocytes to elongating spermatids formed a continuum, which recapitulated the known developmental trajectory.




