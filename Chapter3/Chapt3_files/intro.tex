%!TEX root = ../chapter3.tex
%******************************
%	 Introduction 
%*****************************

\section{Introduction}

\todo{EXPAND INTRODUCTION} 

Sexual reproduction in eukaryotes drives evolution and adaptation \citep{McDonald2016}. It requires the generation of haploid gametes that upon fusion combine their genetic material and develop into a diploid organism. Gametogenesis is a tightly regulated developmental process that differs dramatically between males and females to generate sperm and eggs \citep{Spiller2017}.\\

In spermatogenesis, spermatogonial stem cells undergo a unidirectional differentiation programme to form mature spermatozoa. Spermatogenesis occurs in the epithelium of seminiferous tubules in the testis and is tightly coordinated to ensure the continuous production of mature sperm cells throughout the reproductive lifespan of an animal. In the mouse, this differentiation process initiates with the division of a spermatogonial stem cell (SSC or A$_{\textnormal{single}}$) to form first a pair, and then a connected chain, of undifferentiated spermatogonia (A$_{\textnormal{paired}}$ and A$_{\textnormal{aligned}}$) \citep{Oakberg1971, DeRooij1973}. These cells have competency to undergo spermatogonial differentiation, which involves six transit-amplifying mitotic divisions generating A$_{1-4}$, Intermediate, and B spermatogonia, which then give rise to pre-leptotene spermatocytes (pL) \citep{DeRooij2000}. Pre-leptotene spermatocytes then commit to meiosis, a specialised cell division programme that consists of two consecutive cell divisions without an intermediate S phase to produce haploid cells. In contrast to mitosis, meiosis includes programmed DNA double strand break (DSB) formation, homologous recombination, and chromosome synapsis \citep{Marston2004}. To accommodate these events, prophase of meiosis I is extremely prolonged, lasting several days in males, and can be divided into four substages: leptonema (L), zygonema (Z), pachynema (P) and diplonema (D). Following the two consecutive cell divisions, haploid cells known as round spermatids (RS) are produced, and then undergo a complex differentiation programme called spermiogenesis to form mature spermatozoa \citep{Oakberg1956}.  Spermatogenesis takes place in a highly orchestrated fashion, with tubules periodically cycling through twelve epithelial stages defined by the combination of germ cells present \citep{Oakberg1956}. The completion of one cycle takes 8.6 days in the mouse, and the overall differentiation process from spermatogonia to mature spermatozoa requires approximately 35 days \citep{Oakberg1956a}. Thus, four to five generations of germ cells are present within a tubule at any given time. The continuity of this differentiation process and the gradual transitions between spermatogenic cell types have made the isolation and thus the molecular characterisation of individual sub-stages during spermatogenesis difficult.\\

To elucidate the molecular genetics of germ cell development, we have used an unbiased droplet-based single-cell RNA-Sequencing (scRNA-Seq) approach to capture the continuum of spermatogenic cell populations in the adult testis. We used these data to characterize the complex and dynamic transcriptional profile of spermatogenesis at high-resolution. To confidently identify and label cell populations throughout the developmental trajectory, we profiled cells from juvenile testes during the first wave of spermatogenesis. In juveniles, spermatogenesis has only progressed to a defined developmental stage, and therefore allowed us to unambiguously identify the most mature cell type by comparison with adult. Furthermore, by profiling juvenile samples in which the cell type composition within tubules is heavily biased towards somatic cells and spermatogonia, we obtain the molecular signatures of poorly-characterized cell populations. For spermatogonia, this allows us to visualise and profile their dynamic differentiation process at high resolution. Similarly, we dissected the gene expression heterogeneity within spermatocytes and spermatids, the cell types undergoing meiosis and spermiogenesis, respectively. Our analyses revealed a large number of genes with previously described roles in spermatogenesis, but also allowed us to uncover novel genes that are likely to play important roles during germ cell development and thus relevant for male fertility. \\

Finally, we focused our attention on the inactivation and reactivation of the X chromosome, which is subject to transcriptional silencing as a consequence of asynapsis \citep{Turner2007}. By combining bulk and single-cell RNA-Seq approaches, we identified genes that show \textit{de novo} activation in post-meiotic spermatids with distinct temporal expression patterns. Our data revealed that \textit{de novo} activated escape genes carry distinct chromatin signatures with high levels of repressive H3K9me3 in spermatocytes. Overall, our study presents an in-depth characterization of mouse spermatogenesis and provides new insights into the epigenetic regulation of X chromosome reactivation in post-meiotic spermatids.