%!TEX root = ../main.tex
%******************************
%	 Chapter 2
%*****************************

\chapter{Addressing the mean-variability dependency in scRNA-Seq data}  

\graphicspath{{"../../Dropbox (Cambridge  University)/Figures_for_thesis/Chapter2/"}}

\vfill

\begin{Abstract}
\hspace{-5mm} As shown above, cell-to-cell transcriptional variability in otherwise homogeneous cell populations plays an important role in immune activation and increases with age. Single-cell RNA sequencing can characterise this variability in a transcriptome-wide manner. However, the confounding between variability and mean expression estimates hinders meaningful comparison of expression variability between cell populations. To address this problem, we introduce a statistical approach that extends the BASiCS framework to derive a residual measure of variability that is not confounded by mean expression. This measure is used to test changes in variability in parallel to changes in mean expression on a gene-specific level. With this method, we assessed changes in variability for genes responding to CD4\plus{} T cell activation and detect a synchronisation of biosynthetic machinery components. Furthermore, cytokines such as Il2 that support the activation of surrounding cells by paracrine signalling are heterogeneously up-regulated upon immune activation. When profiling more subtle transcriptional changes during CD4\plus{} T cell differentiation, we detect opposing patterns of changes in variability between \textit{Tbx21} and \textit{Cxcr5}, which are markers for Th1 and Tfh cells, indicating a delayed commitment process throughout differentiation. Finally, we confirmed the applicability of the newly extended BASiCS model to droplet-based scRNA-Seq data which is necessary for the subsequent chapter.
\end{Abstract}

\vfill

\newpage

\begin{Comment}
\hspace{-3mm} \textbf{Declaration} I worked on this project in close collaboration with Catalina Vallejos, Arianne Richard, Sylvia Richardson and John Marioni. In this project, I extended an existing Bayesian framework (BASiCS) that was developed to model scRNA-Seq data. Besides calculating the underlying math and programming the model, I helped re-write the BASiCS R package, which is now published on \href{https://bioconductor.org/packages/release/bioc/html/BASiCS.html}{Bioconductor}. Catalina Vallejos co-supervised me and introduced me to principles of Bayesian statistics. Arianne Richard helped with the interpretation of biological data and helped writing the manuscript. Sylvia Richardson provided statistical help on a part of the project that is not presented here. John Marioni co-supervised me and suggested analysis to be performed. Catalina Vallejos, John Marioni and I designed the study. Catalina Vallejos, John Marioni and I wrote the manuscript. I kindly thank Dominic Gr\"un for providing the smFISH data matched to mESC scRNA-Seq data. The study has been published as:\\

Nils Eling, Arianne C. Richard, Sylvia Richardson, John C. Marioni, Catalina A. Vallejos. Robust expression variability testing reveals heterogeneous T cell responses. \emph{Cell Systems}, In press, 2018 
\end{Comment}

\begin{figure}[hb]
\centering    
\includegraphics[width=0.5\textwidth]{GraphicalAbstract.png}
\end{figure}

\newpage

% Include different main sections of the first chapter
%!TEX root = ../chapter2.tex
%******************************
%	 Introduction 
%*****************************

\section{Introduction}

Heterogeneity in gene expression within a population of single cells can arise from a variety of factors. Structural differences in gene expression within a cell population can reflect the presence of sub-populations of functionally different cell types \citep{Zeisel2015, Paul2015}. Alternatively, in a seemingly homogeneous population of cells,  unstructured expression heterogeneity can be linked to intrinsic or extrinsic noise \citep{Elowitz2002}. Changes in physiological cell states (such as cell cycle, metabolism, abundance of transcriptional/translational machinery and growth rate) represent extrinsic noise, which has been found to influence expression variability within cell populations \citep{Keren2015, Buettner2015, Zeng2017}. Intrinsic noise can be linked to epigenetic diversity \citep{Smallwood2014}, chromatin rearrangements \citep{Buenrostro2015}, as well as the genomic content of single genes, such as the presence of TATA-box motifs and the abundance of nucleosomes around the transcriptional start site \citep{Hornung2012}.  \\ 

Single-cell RNA sequencing (scRNAseq) generates transcriptional profiles of single cells, allowing the study of cell-to-cell heterogeneity on a transcriptome-wide \citep{Grun2014} and single gene level \citep{Goolam2016}. Consequently, this technique can be used to study unstructured cell-to-cell variation in gene expression within and between homogeneous cell populations (i.e.~where no distinct cell sub-types are present). As shown above, transcriptional noise decreases during immune activation. Ageing on the other hand destabilizes immune response which manifests itself in the form of increased transcriptional noise. Furthermore, increasing evidence suggests that this heterogeneity plays an important role in normal development \citep{Chang2008} and that control of expression noise is important for tissue function \citep{BaharHalpern2015}. For instance, molecular noise was shown to increase before cells commit to lineages during differentiation \citep{Mojtahedi2016}, while the opposite is observed once an irreversible cell state is reached \citep{Richard2016}. A similar pattern occurs during gastrulation, where expression noise is high in the uncommitted inner cell mass compared to the committed epiblast and where an increase in heterogeneity is observed when cells exit the pluripotent state and form the uncommitted epiblast \citep{Mohammed2017}. \\

Motivated by scRNA-Seq, recent studies have extended traditional differential expression analyses to explore more general patterns that characterise differences between cell populations \citep[e.g.~][]{Korthauer2016}. As described in the first chapter, BASiCS \citep{Vallejos2015BASiCS,Vallejos2016} introduced a probabilistic tool to assess differences in cell-to-cell heterogeneity between two or more cell populations. To meaningfully assess changes in biological variability across the entire transcriptome, one strong confounding effect must be taken into account: differential variability between populations that is driven by changes in mean expression. This arises because biological noise is negatively correlated with protein abundance \citep{Bar-Even2006, Newman2006, Taniguchi2011} or mean RNA expression (see \textbf{Section \ref{sec0:BASiCS}} and \citep{Brennecke2013, Antolovic2017}). To acknowledge the variance-mean relationship, so far, BASiCS restricts differential variability testing to those genes with equal mean expression across populations (see \textbf{Section \ref{sec1:BASiCS}}). \\


Previous attempts have been made to derive a measure of transcriptional variability which is independent of mean expression. These approaches ranged from a simple linear regression between $\log_2$(CV) and $\log_2$(mean expression) \citep{Wu2017} to more elaborate models as in Gr\"un \emph{et al.}. Here, the aim is to generate a model that captures Poissonian sampling noise for lowly expressed transcripts and increased efficiency noise (differences in total transcript abundance between cells) for highly expressed genes. The mixture of these effects introduces a non-linear relationship between mean expression and the CV. The authors described a model that incorporates reads coming from technical spike-in RNA to capture technical noise in the data. In this case, the number of transcripts available for sequencing is Gamma distributed due to variation in capture efficiency. The sequencing process on the other hand is a Poisson process. The combination of these distribution forms a negative binomial which models the data best \cite{Grun2014}. A non-parmateric strategy to model the mean-variance relationship was proposed in Kolodziejczyk \emph{et al.} where the mean-independent measure of variability is the distance between the CV$^2$ and a roling median along mean expression \citep{Kolodziejczyk2015cell}.\\

In this chapter, we extend the statistical model in BASiCS by implementing a more general approach to account for the confounding effect. By incorporating a flexible, non-linear regression trend, we derive a residual measure of cell-to-cell transcriptional variability that is not confounded by mean expression. This is used to define a probabilistic rule to robustly highlight changes in variability, even for differentially expressed genes. Unlike previous approaches that derive point estimates of residual variability, our approach directly performs gene-specific statistical testing between two conditions using a readily available measure of uncertainty. \\

Using our approach, we identify a synchronisation of  biosynthetic machinery components in CD4\plus{} T cells upon early immune activation as well as an increased variability in the expression of genes related to CD4\plus{} T cell immunological function.
Furthermore, we detect evidence of early cell fate commitment of CD4\plus{} T cells during malaria infection characterized by a decrease in \textit{Tbx21} expression heterogeneity and a rapid collapse of global transcriptional variability after infection. These results highlight biological insights into T cell activation and differentiation that are only revealed by jointly studying changes in mean expression and variability.

\newpage
%!TEX root = ../chapter2.tex
%******************************
%	 Results 
%*****************************

\section{The extended BASiCS model}
\subsection*{Addressing the mean confounding effect for differential variability testing}

\newpage
%!TEX root = ../chapter2.tex
%******************************
%	 Discussion 
%*****************************

\section{Discussion}

In recent years, the importance of modulating cell-to-cell transcriptional variation within cell populations for tissue function maintenance and development has become apparent \citep{BaharHalpern2015, Mojtahedi2016, Goolam2016}. Here, we present a statistical approach to robustly test changes in expression variability between cell populations using scRNA-Seq data. Our method uses a hierarchical Bayes formulation to extend the BASiCS framework by addressing (increasingly popular) experimental protocols where spike-in sequences are not available and by incorporating an additional set of residual over-dispersion parameters $\epsilon_i$ that are not confounded by changes in mean expression. Together, these extensions ensure a broader applicability of the BASiCS software and allow statistical testing of changes in variability that are not confounded by technical noise or mean expression.  \\ 

In general, stable gene-specific variability estimates ideally require a large and deeply sequenced dataset containing a homogeneous cell population \citep[the use of unique molecular identifiers for quantifying transcript counts can also improve variability estimation, see][]{Grun2014}. However, we observe that the regression BASiCS model leads to more stable inference that requires fewer cells to accurately estimate gene-specific summaries, particularly for lowly expressed genes. Despite this, careful considerations should be taken in extreme scenarios where the number of cells is small and/or the data is highly sparse (e.g.~droplet based approaches). These features of the data not only affect parameter estimation but also downstream differential testing. For sparse datasets with low numbers of cells, we recommend the use of a stringent minimum tolerance threshold and/or calibrating the test to a low expected false discovery rate (e.g.~1\%) to avoid detecting spurious signals. Moreover, if possible, an internal calibration can be performed to find a reasonable minimum tolerance threshold (e.g.~by randomly permuting cells between two groups to calibrate the null distribution of the differences between populations). \\

Our method allows characterisation of the extent and nature of variable gene expression in CD4\plus{} T cell activation and differentiation. Firstly, we observe that during acute activation of naive T cells, genes of the biosynthetic machinery are homogeneously up-regulated, while specific immune-related genes become more heterogeneously up-regulated. In particular, increased variability in expression of the apoptosis-inducing Fas ligand \citep{Strasser2009} and the inhibitory ligand PD-L1 \citep{Chikuma2016} suggests a mechanism by which newly activated cells might suppress re-activation of effector cells, thereby dynamically modulating the population response to activation. Likewise, more variable expression of Smad3, which translates inhibitory TGF$\beta$ signals into transcriptional changes \citep{Delisle2013}, may indicate increased diversity in cellular responses to this signal. Increased variability in \textit{Pou2f2} (Oct2) expression after activation suggests heterogeneous activities of the NF-$\kappa$B and/or NFAT signalling cascades that control its expression \citep{Mueller2013}.
Moreover, we detect up-regulated and more variable \textit{Il2} expression, suggesting heterogeneous IL-2 protein expression, which is known to enable T cell population responses \citep{Fuhrmann2016}. \\

Finally, we studied changes in gene expression variability during CD4\plus{} T cell differentiation towards a Th1 and Tfh cell state over a 7 day time course after \textit{in vivo} malaria infection \citep{Lonnberg2017}. Our analysis provides several insights into this differentiation system. Firstly, we observe a tighter regulation in gene expression among genes that do not change in mean expression during differentiation at day 4 at which divergence of Th1 and Tfh differentiation was previously identified \citep{Lonnberg2017}. This decrease in variability on day 4 is potentially due to induction of a strong pan-lineage proliferation program. However, we observe that not all genes follow this trend and uncover four different patterns of variability changes. Secondly, we observe that several Tfh and Th1 lineage-associated genes change in expression variability between days 2 and 4. For example, we noted a decrease in variability for one key Th1 regulator, \textit{Tbx21} (encoding Tbet), which suggests that a subset of cells may have already committed to the Th1 lineage at day 2. Three additional Th1 lineage-associated genes also followed this trend (\textit{Ahnak}, \textit{Ctsd}, \textit{Tmem154}). These data suggest that differentiation fate decisions may arise as early as day 2 in subpopulations within this system, resulting in high gene expression variability. Such an effect is in accordance with the early commitment to effector T cell fates that was previously observed during viral infection \citep{Choi2011}. As these results illustrate, diversity in differentiation state within a population of T cells can drive our differential variability results. To further disect these results, subsequent analyses such as the pseudotime inference used in \cite{Lonnberg2017} could be used to characterize a continuous differentiation process.\\

In sum, our model provides a robust tool for understanding the role of heterogeneity in gene expression during cell fate decisions. With the increasing use of scRNA-Seq to study this phenomenon, our and other related tools will become increasingly important.




